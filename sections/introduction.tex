\section{Introduction}
The rule of the logic-based Sudoku game includes filling the grid with numbers from 1 to $n$ without repetition to row, column, and subgrid. A valid Sudoku solution must simultaneously satisfy multiple interdependent constraints which makes it a good benchmark for constraint satisfaction problems~\cite{lloyd2020,simonis2005sudoku}. Backtracking is the popular traditional way of solving Sudoku~\cite{norvig2006solving,Schott,bhattarai2025study}. It tries to solve the puzzle by filling every cell with valid values then goes back to the previous assignment if it leads to invalid puzzle. It was then followed by heuristic approaches~\cite{crook2009pencil,bhattarai2025study} who mimicked human reasoning in solving Sudoku. However, these approaches suffered from exponential increase of time. To overcome this limitation, metaheuristic approaches specifically swarm-based algorithms emerged. Artificial Bee Colony (ABC)~\cite{pacurib2009}, Ant Colony Optimization (ACO)~\cite{lloyd2020}, and Particle Swarm Optimization (PSO)~\cite{PSO, GPSO} are a few examples of these approaches. Though ABC’s performance showed potential by simulating foraging behaviour, it still struggled on more complex Sudoku puzzles. Unlike PSO and ABC, ACO was able to provide solutions with less execution time thereby indicating that the convergence speed in ACO is faster. Due to its potential, a multi-colony ant optimization was designed to test its performance when there are multiple colonies of ants in the search space. The Dynamic Collaborative Multi-Colony Ant Optimization (DCM-ACO) allows sharing of pheromone information between multiple colonies which improves diversity and convergence speed thus outperforming the single colony~\cite{mo2022}. 
Parallelization of swarm-based algorithms was developed as an effort to improve the performance of the algorithms. One of these is the Parallel Independent Runs (PIR) but no sharing of information between independent agents leads to redundant exploration in the search space ~\cite{yang2015}. There is another parallelization technique that focuses on ants which leads to synchronization overhead when a single ant updates global pheromone~\cite{randall2002}. 

To address the exponential increase in time and stagnation, this paper proposes SudoSLVRR, a multithreaded sudoku solver that combines constraint propagation and DCM-ACO within threads that communicate according to the communication topologies. SudoSLVRR is designed to balance exploration and exploitation which improves scalability on large-scale Sudoku puzzles. 

